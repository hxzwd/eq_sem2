
\documentclass[12pt,a4paper,draft]{letter}
\usepackage[utf8]{inputenc}
\usepackage[russian]{babel}
\usepackage[OT1]{fontenc}
\usepackage{cmap}
\usepackage{amsmath}
\usepackage{amsfonts}
\usepackage{amssymb}
\usepackage{setspace} 
\usepackage[left=2cm,right=2cm,top=2cm,bottom=2cm]{geometry}
\begin{document}
\onehalfspacing
Уравнение для эллиптических функций Вейерштрасса (1):
\\
$E_R [R(z)] = \left(\dfrac{dR(z)}{dz}\right)^2 - 4R^3(z) - aR^2(z) - bR(z) - c = 0$
\\
Уравнение Риккати (2):
\\
$E_Y [Y(z)] = \dfrac{dY(z)}{dz} + Y^2(z) - \beta = 0$
\\
Хотим выразить решения $E_R [R(z)] = 0$ через решения уравнения $E_Y [Y(z)] = 0$
\\
Порядок полюса уравнения $E_R [R(z)] = 0$:
\\
$p = 2$
\\
Тогда решение $E_R [R(z)] = 0$ выражается через решения уравнения $E_Y [Y(z)] = 0$
\\
следующим образом:
\\
$R(z) = A_0 + A_1 Y(z) + A_2 Y^2(z)$
\\
Требуется найти коэффициенты: $A_0, A_1, A_2$
\\
Выполняем подстановку $R(z) = A_0 + A_1 Y(z) + A_2 Y^2(z)$ в уравнение $E_R [R(z)] = 0$.
\\
После подстановки член уравнения (1) $\left(\dfrac{dR(z)}{dz}\right)^2$ породит производную $ \dfrac{dY(z)}{dz}$ функии $Y(z)$.
\\
Выразим из уравнения (2) проивзодную $ \dfrac{dY(z)}{dz}$ и подставим в соответствующее выражение.
\\
Получим полином шестой степени относительно функции $Y(z)$:
\\
$P(Y(z), A_0, A_1, A_2, a, b, c, \beta) = 0$
\\
Для того, чтобы он был равен нулю необходимо, чтобы были равны нулю коэффициенты при всех степенях $Y(z)$.
\\
Выпишем коэффициенты при разных степенях $Y(z)$ (система уравнений (3)):
\\
$Y^6(z): -4A_2^3 + 4A_2^2 = 0$ \\
$Y^5(z): -12A_1 A_2^2 + 4A_1 A_2 = 0$ \\
$Y^4(z): -12 A_0 A_2^2 - 12 A_1^2 A_2 + A_1^2 - A_2^2 a - 8 A_2^2 \beta = 0$ \\
$Y^3(z):  -24 A_0 A_1 A_2 - 4 A_1^3 - 2 A_1 A_2 a - 8 A_1 A_2 \beta = 0$ \\
$Y^2(z):  -12 A_0^2 A_2 - 12 A_0 A_1^2 - 2 A_0 A_2 a - A_1^2 a - 2 A_1^2 \beta + 4 A_2^2 \beta^2 - A_2 b = 0$ \\
$Y(z):  -12 A_0^2 A_1 - 2 A_0 A_1 a + 4 A_1 A_2 \beta^2 - A_1 b = 0$ \\
$Y^0(z):  -4 A_0^3 - A_0^2 a - A_0 b + A_1^2 \beta^2 - c = 0$ \\
Из первого уравнения очевидно, что $A_2 = 1$ \\
Подставим $A_2 = 1$ в оставшиеся уравнения, получим более простую систему следующего вида (система уравнений (4)): 
\\
$Y^5(z): -8 A_1= 0$\\
$Y^4(z): -12 A_0 - 11 A_1^2 - a - 8 \beta= 0$\\
$Y^3(z):  -2 A_1 (12 A_0 + 2 A_1^2 + a + 4 \beta)= 0$\\
$Y^2(z):  -12 A_0^2 - 12 A_0 A_1^2 - 2 A_0 a - A_1^2 a - 2 A_1^2 \beta - b + 4 \beta^2= 0$\\
$Y(z):  A_1 (-12 A_0^2 - 2 A_0 a - b + 4 \beta^2)= 0$\\
$Y^0(z):  -4 A_0^3 - A_0^2 a - A_0 b + A_1^2 \beta^2 - c= 0$\\
Из первого уравнения новой системы очевидно, что $A_1 = 0$\\
Аналогично предыдушему шагу, подставим значение $A_1 = 0$ в последнюю систему уравнений.\\
Из второго уравнения, соответствующему коэффициенту при $Y^4(z)$, выразим значение $A_0$.\\
Получим:\\
$A_0 = -\frac{a}{12} - \frac{2 \beta}{3}$\\
Выполним подстановку коэффициентов $A_0$ и $A_1$ в уравнения системы (4).
Получим новую систему из двух уравнений, следующего вида (система уравнений (5)):\\
$Y^2(z): \frac{a^2}{12} - b - \frac{4 \beta^2}{3} = 0$\\
$Y^0(z): -\frac{a^3}{216} - \frac{a^2 \beta}{18} + \frac{a b}{12} + \frac{2 b \beta}{3} + \frac{32 \beta^3}{27} - c = 0$\\
Полученная система уравнений показывает связь между параметрами $a, b, c, \beta$\\
Например, пусть в исходной задаче, коэффициенты $a$ и $b$ уравнения (1) имеют значения:\\
$a = 4$\\
$b = 1$\\
Тогда, после подставновки этих значений в систему уравнений (5), получим значения для параметров $\beta$ и $c$:\\
$\beta = \frac{1}{2}$\\
$c = \frac{2}{27}$\\
Таким образом, специальные решения уравнения:
$E_R [R(z)] = \left(\dfrac{dR(z)}{dz}\right)^2 - 4R^3(z) - 4 R^2(z) - R(z) - \frac{2}{27} = 0$\\
Выражаются через решения уравнения Риккати вида:
$E_Y [Y(z)] = \dfrac{dY(z)}{dz} + Y^2(z) - \frac{1}{2} = 0$\\
Следующим образом:\\
$R(z) = -\frac{2}{3} + Y^2(z)$\\                                                                  
\end{document}



