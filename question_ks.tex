
\documentclass[12pt,a4paper,draft]{letter}
\usepackage[utf8]{inputenc}
\usepackage[russian]{babel}
\usepackage[OT1]{fontenc}
\usepackage{cmap}
\usepackage{amsmath}
\usepackage{amsfonts}
\usepackage{amssymb}
\usepackage{setspace} 
\usepackage[left=2cm,right=2cm,top=2cm,bottom=2cm]{geometry}
\begin{document}
\onehalfspacing
Уравнение Курамото-Сивашинского (0):
\\
$u_t + u u_x + \alpha u_{xx} + \beta u_{xxx} + \gamma u_{xxxx} = 0$\\
Уменьшим количество параметров уравнения (0) с помощью следующих замен переменных:\\
$u = u' \alpha \sqrt{\frac{\alpha}{\gamma}}$\\
$x = x' \sqrt{\frac{\gamma}{\alpha}}$\\
$t = t' \sqrt{\frac{\gamma^2}{\alpha}}$\\
$\sigma = \frac{\beta}{\sqrt{\alpha \gamma}}$\\
Опуская штрихи, получим уравнение с одним параметром \sigma:\\
$u_t + u u_x + u_{xx} + \sigma u_{xxx} + u_{xxxx} = 0$\\
Будем искать решение полученного уравнения в переменных бегущей волны, тогда:\\
$z = x - C_0 t$\\
$u = u(x, t) = y(x - C_0 t) = y(z)$\\
Получим уравнение относительно функции $y(z)$ следующего вида:\\
$-C_0 y_z + y y_z + y_{zz} + \sigma y_{zzz} + y_{zzzz} = 0$\\
Проинтегрируем уравнение выше, получим уравнение вида (обозначим его как уравнение (1)):\\
$E_y [y(z)] = \frac{1}{2} y^2 + y_{zzz} + \sigma y_{zz} + y_z - C_0 y + C_1 = 0$\\
Уравнение Риккати (2):
\\
$E_Y [Y(z)] = \dfrac{dY(z)}{dz} + Y^2(z) - b = 0$
\\
Хотим выразить решения $E_y [y(z)] = 0$ через решения уравнения $E_Y [Y(z)] = 0$
\\
Порядок полюса уравнения $E_y [y(z)] = 0$:
\\
$p = 3$
\\
Тогда решение $E_R [R(z)] = 0$ выражается через решения уравнения $E_Y [Y(z)] = 0$
\\
следующим образом:
\\
$y(z) = A_0 + A_1 Y(z) + A_2 Y^2(z) + A_3 Y^3(z)$
\\
Требуется найти коэффициенты: $A_0, A_1, A_2, A_3$
\\
Выполняем подстановку $y(z) = A_0 + A_1 Y(z) + A_2 Y^2(z) + A_3 Y^3 (z)$ в уравнение $E_y [y(z)] = 0$.
\\
После подстановки выражения для $y(z)$ в уравнение (1) будут порождены производные функции $Y(z)$ до\\
третьего порядка включительно.\\
Выполним для производных функции $Y(z)$ соответствующие подстановки, полученные из уравнения (2):\\
$\dfrac{dY(z)}{dz} = a Y(z) + b - Y^2(z)$\\
$\dfrac{d^2 Y(z)}{dz^2} = a^2 Y(z) + a b - 3 a Y^2(z) - 2 b Y(z) + 2 Y^3(z)$\\
$\dfrac{d^3 Y(z)}{dz^3} = a^3 Y(z) + a^2 b - 7 a^2 Y^2(z) - 8 a b Y(z) + 12 a Y^3(z) - 2 b^2 + 8 b Y^2(z) - 6 Y^4(z)$\\
В приведенных подстановках примем во внимание, что параметр $a = 0$.\\
член уравнения (1) $\left(\dfrac{dR(z)}{dz}\right)^2$ породит производную $ \dfrac{dY(z)}{dz}$ функии $Y(z)$.
В конечном итоге Получим полином шестой степени относительно функции $Y(z)$:
\\
$P(Y(z), A_0, A_1, A_2, A_3, \sigma, b, C_0, C_1) = 0$
\\
Для того, чтобы он был равен нулю необходимо, чтобы были равны нулю коэффициенты при всех степенях $Y(z)$.
\\
Выпишем коэффициенты при разных степенях $Y(z)$ (система уравнений (3)):
\\
$Y^6(z): 0.5 A_3^2 - 60.0 A_3 = 0$\\
$Y^5(z): 1.0 A_2 A_3 - 24.0 A_2 + 12.0 A_3 \sigma = 0$\\
$Y^4(z): 1.0 A_0 A_1 A_3 - 6.0 A_0 A_1 + 0.5 A_2^2 + 6.0 A_2 \sigma + 114.0 A_3 b - 3.0 A_3 = 0$\\
$Y^3(z):  1.0 A_0 A_1 A_2 + 2.0 A_0 A_1 \sigma + 40.0 A_2 b - 2.0 A_2 - 1.0 A_3 C_0 - 18.0 A_3 b \sigma = 0$\\
$Y^2(z): 0.5 A_0^2 A_1^2 + 8.0 A_0 A_1 b - 1.0 A_0 A_1 - 1.0 A_2 C_0 - 8.0 A_2 b \sigma - 60.0 A_3 b^2 + 3.0 A_3 b = 0$\\
$Y(z): -1.0 A_0 A_1 C_0 - 2.0 A_0 A_1 b \sigma - 16.0 A_2 b^2 + 2.0 A_2 b + 6.0 A_3 b^2 \sigma = 0$\\
$Y^0(z): -2.0 A_0 A_1 b^2 + 1.0 A_0 A_1 b + 2.0 A_2 b^2 \sigma + 6.0 A_3 b^3 + 1.0 C_1 = 0$\\

Из первого уравнения очевидно, что $A_3 = 120$ \\
Подставим $A_3 = 120$ в оставшиеся уравнения, получим более простую систему следующего вида (система уравнений (4)): \\
$Y^5(z): 96.0 A_2 + 1440.0 \sigma = 0$\\
$Y^4(z): 114.0 A_0 A_1 + 0.5 A_2^2 + 6.0 A_2 \sigma + 13680.0 b - 360.0 = 0$\\
$Y^3(z): 1.0 A_0 A_1 A_2 + 2.0 A_0 A_1 \sigma + 40.0 A_2 b - 2.0 A_2 - 120.0 C_0 - 2160.0 b \sigma = 0$\\
$Y^2(z): 0.5 A_0^2 A_1^2 + 8.0 A_0 A_1 b - 1.0 A_0 A_1 - 1.0 A_2 C_0 - 8.0 A_2 b \sigma - 7200.0 b^2 + 360.0 b = 0$\\
$Y(z): -1.0 A_0 A_1 C_0 - 2.0 A_0 A_1 b \sigma - 16.0 A_2 b^2 + 2.0 A_2 b + 720.0 b^2 \sigma = 0$\\
$Y^0(z): -2.0 A_0 A_1 b^2 + 1.0 A_0 A_1 b + 2.0 A_2 b^2 \sigma + 1.0 C_1 + 720.0 b^3 = 0$\\
Из первого уравнения новой системы выразим $A_2$ через $\sigma$:\\
$A_2 = -15 \sigma$\\
Подставим выражение для $A_2$ во все уравнения системы (4), получим систему уравнений (5):\\
$Y^4(z): 114.0 A_0 A_1 + 13680.0 b + 22.5 \sigma^2 - 360.0 = 0$\\
$Y^3(z): -13.0 A_0 A_1 \sigma - 120.0 C_0 - 2760.0 b \sigma + 30.0 \sigma = 0$\\
$Y^2(z): 0.5 A_0^2 A_1^2 + 8.0 A_0 A_1 b - 1.0 A_0 A_1 + 15.0 C_0 \sigma - 7200.0 b^2 + 120.0 b \sigma^2 + 360.0 b = 0$\\
$Y(z): -1.0 A_0 A_1 C_0 - 2.0 A_0 A_1 b \sigma + 960.0 b^2 \sigma - 30.0 b \sigma = 0$\\
$Y^0(z): -2.0 A_0 A_1 b^2 + 1.0 A_0 A_1 b + 1.0 C_1 + 720.0 b^3 - 30.0 b^2 \sigma^2 = 0$\\
Как решать системы подобного рода?\\
Как проверить есть ли у них вообще решение.\\
В данной системе 5 уравнений и 6 неизвестных: $A_0, A_1, b, C_1, C_0, \sigma$.\\
Часть этих неизвестных, а именно: $C_1, C_0, \sigma$ относятся к входным данным самой задачи.\\
То есть мы не можем использовать метод простейших уравнений для произволного выбранных\\
значений параметров: $\sigma, C_0, C_1$, которые относятся непосредственно к уравнению\\
выского порядка (уравнение Курамото-Сивашинского в переменных бегущей волны), которое\\
мы пытаемся решать? (эти параметры будут ограничены множеством решений системы уравнений (5))\\     
\end{document}